\documentclass[journal,12pt, two column]{IEEEtran}
\usepackage{mathtools}
\usepackage{amsmath}
\usepackage{enumitem}
\usepackage{graphicx}
\title{Assignment 2\\ \Large AI1110: Probability and Random Variables \\ \large Indian Institute of Technology Hyderabad}
\author{Shreyas Wankhede \\ \normalsize AI21BTECH11028 \\ \vspace*{20pt} \\ \Large ICSE 2017 Grade 12}



\begin{document}

\maketitle

\textbf{Question (1.X) :}
solve the equation:
\begin{center}
$\dfrac{dy}{dx} =1 - xy + y - x$\\
\end{center}

\textbf{Solution:}
\begin{align}
 \dfrac{dy}{dx} =1 - xy + y - x
\end{align}

\textbf{Some standard results used:}
\begin{enumerate}[label=\roman*]
\item $\int x^n dx = \frac{x^{n+1}}{n+1} + c$ \label{eq:i}
\item $\int \frac{1}{x}\ dx = \log_e x + c$ \label{eq:ii}
\end{enumerate}

Using Variable Separable method,
\begin{align}
&\implies\dfrac{dy}{dx} =1 - x + y(1-x)\\
&\implies\dfrac{dy}{dx} =(1-x)(1+y)\\
&\implies\dfrac{dy}{(1+y)} =(1-x)dx
\end{align}

Integrating both sides,
\begin{align}
&\implies\int\dfrac{dy}{(1+y)} =\int(1-x)dx\label{eq:5}
\end{align}

Using formula \eqref{eq:ii} for LHS,
\begin{align}
\implies\int\dfrac{dy}{(1+y)}=\log_e\mid1+y\mid+c\label{eq:6}
\end{align}

Using formula \eqref{eq:i} for RHS,
\begin{align}
\implies\int(1-x)dx=x - \dfrac{x^2}{2} + c\label{eq:7}
\end{align}

From \eqref{eq:5}, \eqref{eq:6} and \eqref{eq:7},
\begin{align}
\implies\log_e\mid1+y\mid = x - \dfrac{x^2}{2} + c\label{eq:8}
\end{align}

where 'c' is the integration constant.


\begin{figure}[!ht]
		\centering
		\includegraphics[width=\columnwidth]{figure_1.png}
		\caption{graph for curve corresponding to equation \eqref{eq:8}.(where integration constant c is assumed to be 0}
		\label{fig-1}
	\end{figure}

\end{document}